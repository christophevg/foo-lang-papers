\documentclass[conference]{IEEEtran}
\usepackage[pdftex]{graphicx}
\usepackage[cmex10]{amsmath}
\interdisplaylinepenalty=2500
\usepackage{algorithmic}
\usepackage{array}
\usepackage{mdwmath}
\usepackage{mdwtab}
\usepackage{eqparbox}
\usepackage[font=footnotesize]{subfig}
\usepackage{fixltx2e}
\usepackage{url}
\hyphenation{op-tical net-works semi-conduc-tor}

% additional packages and utility commands
\usepackage{minted}
% \usepackage{flushend}

% \usepackage{lineno}
% \setlength\linenumbersep{1mm}
% \linenumbers

\begin{document}

% to avoid syntax error highlighting - e.g. foo's not js ;-)
\expandafter\def\csname PY@tok@err\endcsname{}

\title{Efficiently Detecting Intrusions\\in the Internet of Things}

\author{\IEEEauthorblockN{Christophe Van Ginneken, Jef Maerien, Christophe
Huygens, Danny Hughes, Wouter Joosen}%
\IEEEauthorblockA{iMinds-DistriNet, KU Leuven\\
3001 Leuven, Belgium\\
\{firstname.lastname\}@cs.kuleuven.be}}

\maketitle

\begin{abstract}

Supporting multiple intrusion detection (ID) algorithms imposes significant
overhead on Internet of Things (IoT) devices. Fusion of the underlying
algorithms can alleviate this, yet, proves to be time-consuming, repetitive and
error-prone. To address this problem we introduce id-foo, a framework for the
development of efficient intrusion detection systems (IDS). It consists of a
domain specific language (DSL) that allows describing ID algorithms in a formal
and functional way. The id-foo code generator then fuses these algorithms on a
functional level and produces more optimally organised source code. A
side-by-side comparison shows that id-foo-based code significantly reduces
memory footprint, message passing overhead and execution time in comparison to
a straightforward and sequential implementation of the same algorithms.

\end{abstract}

\section{Introduction}

WSN \cite{baggio2005wireless,werner2005monitoring,hughes2006gridstix}

Economics of development, IDS as a non-functional cost \cite{lee2002toward}

Energy in WSNs \cite{hughes2013energy}

Functional Code Fusion

The remainder of this paper proceeds as follows: ...

\section{Related Work}

IDS in WSN \cite{perrig2004security,mishra2004intrusion}

DSL \cite{fowler2010domain,mernik2005and}

DSL for WSN \cite{naumowicz2009prototyping,levis2004tinyscript}

DSL for IDS \cite{eckmann2002statl}

Code Generation for embedded systems/WSNs \cite{leupers2000code,marwedel2002code}

Code Optimization for embedded systems/WSNs \cite{panda2001data,naik2001software}

Code Generation for IDS \cite{charitakis2003code}

\section{Background}

Theory of Loop Fusion \cite{darte2000complexity}

Examples of code/loop fusion (e.g. Haskell Stream Fusion) \cite{coutts2007stream}

Function Inlining (e.g. TinyOS) \cite{gay2007software,gay2003nesc}

\section{Problem Analysis}

TODO

\section{Problem Solution}

TODO

\section{Evaluation}

network usages: number of packets, number of bytes

processing time

image size

TODO: memory usage ?

2 algorithms \cite{ganeriwal2008reputation}

TODO: add 3rd ? \cite{krontiris2009cooperative}

\section{Conclusions and Future Work}

theoretical gains are met in practice

explore more domains + implement corresponding DSLs for embedded systems

extract generic language as host for DSLs

\section*{Acknowledgements}

This research is partially funded by the Interuniversity Attraction Poles
Programme Belgian State, Belgian Science Policy, and by the Research Fund KU
Leuven. This research is partially funded by the EU FP7 project NESSoS. We
would like to thank the reviewers for their thoughtful and helpful comments
that enhanced the readability of this paper. Our gratitude and respect also
goes out to all members of the NES task force at KU Leuven/DistriNet for
creating the nurturing environment where these ideas could grow.

\bibliographystyle{IEEEtran}
\bibliography{literature/referenties}

\end{document}
