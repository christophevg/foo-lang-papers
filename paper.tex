\documentclass[conference]{IEEEtran}
\usepackage[pdftex]{graphicx}
\usepackage[cmex10]{amsmath}
\interdisplaylinepenalty=2500
\usepackage{algorithmic}
\usepackage{array}
\usepackage{mdwmath}
\usepackage{mdwtab}
\usepackage{eqparbox}
\usepackage[font=footnotesize]{subfig}
\usepackage{fixltx2e}
\usepackage{url}
\hyphenation{op-tical net-works semi-conduc-tor}

% additional packages and utility commands
\usepackage{minted}
% \usepackage{flushend}

% \usepackage{lineno}
% \setlength\linenumbersep{1mm}
% \linenumbers

\begin{document}

% to avoid syntax error highlighting - e.g. foo's not js ;-)
\expandafter\def\csname PY@tok@err\endcsname{}

\title{A Domain Specific Language for Functional Code Fusion of
       Intrusion Detection Algorithms in Wireless Sensor Networks}

\author{\IEEEauthorblockN{Christophe Van Ginneken, Jef Maerien, Christophe
Huygens, Danny Hughes, Wouter Joosen}%
\IEEEauthorblockA{iMinds - DistriNet - KU Leuven\\
B-3001, Leuven, Belgium\\
\{firstname.lastname\}@cs.kuleuven.be}}

\maketitle

\begin{abstract}
  
Introducing intrusion detection in a wireless sensor network puts a strain on
the limited resources of its nodes. Fusing the underlying algorithms' code can
alleviate this, but the manual and repetitive labor proves to an unacceptable
economic burden. Intrusion detection algorithms share common functionality and
therefore are a prime candidate for applying a domain specific language. In
this paper we propose such a domain specific language to describe these
algorithms in a formal and functional way. This allows for the automation of
the fusion of these algorithms on a functional level though code generation.
The code, produced by a prototype code generator, shows to have less impact on
e.g. the execution time of the resulting intrusion detection system and its use
of the wireless network.

\end{abstract}

\bibliographystyle{IEEEtran}
\bibliography{referenties}

\end{document}
