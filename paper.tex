\documentclass[conference]{IEEEtran}
\usepackage[pdftex]{graphicx}
\usepackage[cmex10]{amsmath}
\interdisplaylinepenalty=2500
\usepackage{algorithmic}
\usepackage{array}
\usepackage{mdwmath}
\usepackage{mdwtab}
\usepackage{eqparbox}
\usepackage[font=footnotesize]{subfig}
\usepackage{fixltx2e}
\usepackage{url}
\hyphenation{}

% additional packages and utility commands
\usepackage{minted}
\usepackage{flushend}

\usepackage{lineno}
\setlength\linenumbersep{1mm}
\linenumbers

% until we settle on the final name ;-)
\usepackage{xspace}
\newcommand{\NAME}{id-foo\xspace}

\begin{document}

% to avoid syntax error highlighting - e.g. foo's not js ;-)
\expandafter\def\csname PY@tok@err\endcsname{}

\title{
Introducing the \NAME Framework\\
for Efficient Intrusion Detection\\
in the Internet of Things
}

\author{\IEEEauthorblockN{Christophe Van Ginneken, Jef Maerien, Christophe
Huygens, Danny Hughes, Wouter Joosen}%
\IEEEauthorblockA{iMinds-DistriNet, KU Leuven\\
3001 Leuven, Belgium\\
\{firstname.lastname\}@cs.kuleuven.be}}

\maketitle

\begin{abstract}

Supporting multiple intrusion detection (ID) algorithms imposes significant
overhead on Internet of Things (IoT) devices. Fusion of the underlying
algorithms can alleviate this, yet, proves to be time-consuming, repetitive and
error-prone. To address this problem we introduce \NAME, a framework for the
development of efficient intrusion detection systems (IDS). It consists of a
domain specific language (DSL) that allows formally describing ID algorithms on
a functional level. A code generator then fuses these algorithms and produces
more optimally organised source code. This way, \NAME addresses the
heterogeneity of both the IoT and ID, as well as the limited resources of IoT
devices. A side-by-side comparison shows that generated code reduces memory
footprint, message passing overhead and execution time in comparison to a
straightforward sequential implementation of the same algorithms.

\end{abstract}

\section{Introduction}

% context

% IoT + threat

The promise of the Internet of Things (IoT) is exciting: my smart car will
notify the my smart home \cite{aldrich2003smart} that I will be late. It now
can turn up the heat later, thus lowering my energy bills and adding to a
greener world. Wonderful, I'm in. But what if the energy company can do the
same? What if they are able to turn up the heat while I'm gone, making my home
nice and cosy but also cranking up my bill and their profit. No thank you, let
me out.

Along with all the wonderful possibilities, the IoT clearly holds a significant
threat: by opening our homes and lives to all these interconnected devices, we
also open them to everyone who is able to break these new, virtual locks, often
without leaving a trace.

% classic networks: fw and central ids

In classic networks, firewalls protect the outer perimeter of the network,
filtering unwanted packets. But attacks on flaws in services can pass
unnoticed. This is where intrusion detection (ID) steps in: an ID probe
monitors all traffic that passes through the firewall, looking for patterns and
optionally alerting the firewall if a malicious pattern is detected.

% manet: local ids -> localize -> limited resources

In mobile ad hoc networks (MANET), which make up an important part of the IoT,
it is not possible to have this single point in the network that can oversee
all traffic \cite{mishra2004intrusion}. If the outer security perimeter
diminishes, every device has to implement its own lines of defence.
Localisation of the IDS requires resources, and resources are another important
factor to consider: IoT devices typically have limited batteries, little
processing power and memory is restricted to the bare minimum.

% gap analysis

Current developments in ID focus on programming frameworks to structure the
implementation of different algorithms \cite{valero2012di} and offer the
required basic functional components to implement actual algorithms
\cite{krontiris2008lidea}, but don't offer a way to optimise the usage of
resources. They also don't support systematic reuse to create different
configurations for different nodes or evolution in time.

% problem -> requirements for solution

The IoT is a heterogeneous environment, both in hardware and software.
Addressing this in a transparent and automated way is a key success factor for
any kind of development in this area. For a non-functional layer, such as ID,
this aspect becomes even more paramount.

A solution that supports the integration of multiple algorithms on IoT devices
needs to consider the impact on their resources. Different algorithms should be
merged to avoid accumulating their impact.

% my approach merged into ...
% scientific contribution

% 1. framework: language + code generator

The first contribution of this paper is a framework that enables the formal
description of ID algorithms on a functional level. An automated code
generation process produces corresponding source code, for a given platform and
configuration. The way, the framework addresses the heterogeneous nature of
both the IoT as a hardware and software platform, and that of the different ID
algorithms.

% 2. FCF to optimise

The second contribution introduces functional code fusion (FCF) as a generation
paradigm. FCF identifies common data and functions, and organises code in such
a way that redundant iterations, tests and computations are eliminated. Our
results show substantially improved memory usage, execution time and usage of
the wireless radio, reducing energy consumption.

% structure

The remainder of this paper proceeds as follows: section \ref{background}
details some of the ideas behind FCF and identifies the different aspects of
the problem space. Section \ref{design} describes the design we applied to
construct the framework, its domain specific language (DSL) and code generator.
Section \ref{evaluation} evaluates an implementation and application of the
framework. Section \ref{related} explores related work in the field of DSLs for
ID and code generation for the IoT. Finally, section \ref{conclusion}
summarises our findings, draws conclusions and identifies topics for future
work.

\section{Background}
\label{background}

Theory of Loop Fusion \cite{darte2000complexity}

Examples of code/loop fusion (e.g. Haskell Stream Fusion) \cite{coutts2007stream}

Function Inlining (e.g. TinyOS) \cite{gay2007software,gay2003nesc}

TODO

\section{Design}
\label{design}

TODO

\section{Implementation and Evaluation}
\label{evaluation}

network usages: number of packets, number of bytes

processing time

image size

TODO: memory usage ?

2 algorithms \cite{ganeriwal2008reputation}

TODO: add 3rd ? \cite{krontiris2009cooperative}

\section{Related Work}
\label{related}

IDS in WSN \cite{perrig2004security,mishra2004intrusion}

DSL \cite{fowler2010domain,mernik2005and}

DSL for WSN \cite{naumowicz2009prototyping,levis2004tinyscript}

DSL for IDS \cite{eckmann2002statl}

Code Generation for embedded systems/WSNs \cite{leupers2000code,marwedel2002code}

Code Optimization for embedded systems/WSNs \cite{panda2001data,naik2001software}

Code Generation for IDS \cite{charitakis2003code}

% manet: IDS algorithms -> coorporation, trust

More than in a classic network IDS, ID algorithms for MANETS operate with a
large degree of uncertainty. They often have to collaborate with other devices
in the network \cite{marchang2008collaborative,krontiris2009cooperative}, which
increases the level of uncertainty, because no other device can fully be
trusted. An IDS on an IoT device will be able to provide the services with
information about the reputation \cite{ganeriwal2008reputation} of other nodes
in the network.

\section{Conclusion and Future Work}
\label{conclusion}

theoretical gains are met in practice

explore more domains + implement corresponding DSLs for embedded systems

extract generic language as host for DSLs

Integration of applications with generated IDS allows actively and dynamically
responding to changes in the reputation of other nodes.

The services running on those devices can take advantage of these local
security layers. If the IDS detects a problem, the service can immediately
terminate the communication with the attacker, before it's actually abused.

\section*{Acknowledgements}

This research is partially funded by the Interuniversity Attraction Poles
Programme Belgian State, Belgian Science Policy, and by the Research Fund KU
Leuven. This research is partially funded by the EU FP7 project NESSoS. We
would like to thank the reviewers for their thoughtful and helpful comments
that enhanced the readability of this paper. Our gratitude and respect also
goes out to all members of the NES task force at KU Leuven/DistriNet for
creating the nurturing environment where these ideas could grow.

\bibliographystyle{IEEEtran}
\bibliography{literature/referenties}

\end{document}
